\documentclass[sigconf]{acmart}

\usepackage{graphicx}
\usepackage{hyperref}
\usepackage{todonotes}

\usepackage{endfloat}
\renewcommand{\efloatseparator}{\mbox{}} % no new page between figures

\usepackage{booktabs} % For formal tables

\settopmatter{printacmref=false} % Removes citation information below abstract
\renewcommand\footnotetextcopyrightpermission[1]{} % removes footnote with conference information in first column
\pagestyle{plain} % removes running headers

\newcommand{\TODO}[1]{\todo[inline]{#1}}

\begin{document}
\title{Big Data Analytics in Sports - Track and Field}


\author{Mathew Millard}
\affiliation{%
  \institution{Indiana University Bloomington}
  \city{Bloomington} 
  \state{Indiana} 
  \country{USA}
}
\email{mdmillar@indiana.edu}

\renewcommand{\shortauthors}{M. Millard}

\begin{abstract}
Over the years, sports analytics has become more prominent in the way sports are evolving. Data is the driving force in how these improvements and changes are being made. With the presence of Big data and the impact it has on how sports decisions are being made, the importance on how this data can be used and how it can be presented in a useful manner have become a focal point for furthering athletics. This focus on data can be seen all over the professional sports world with many teams hiring data scientists and analysts to use data in order to get the most out of how they develop their rosters. Big data analytics within sports can cover a broad spectrum of topics, but the focus here is to dive deeper into the sport of track and field. First, we will take a look at what track and field looked like without the use of big data. Then, we will dive into the impacts big data has had on the development of track and field.
\end{abstract}

\keywords{i523, hid216, Track and Field, Big Data, Sports, Running}


\maketitle

\section{Introduction}
When people think about big data analytics in sports, many think about sports such as basketball, baseball, and football because of the popularity and the volume of statistics used in those sports. Track and field, however, can benefit greatly from the same treatment that other sports have received. Although the sport of track and field is not a widely popular sport, there are many who take part in the sport due to the accessibility of being involved. Events on the running side of this sport range from the one hundred meter dash all the way up to the longer distance races such as the ten thousand meter run with many different types of races in between. On the field event side, there are various events including long jump, high jump, javelin, and many more. With all of these different events, there is a strong sense of specialization among athletes. An attention to detail and technique in the many facets of the sport is what drives the need for data analytics and statistical analysis.

\section{Track and Field Before Big Data}
Like all sports, track and field had to start somewhere and it was much different before the world of technology and analytics gave way for a much needed boost. In the days before technology became prevalent, statistical analytics were harder to come by and much of what was being done in the sport was based on theory and vague understanding of how the body worked. One of the most notable aspects of the sport that has come a long way since the injection of data analysis is the footwear that the athletes train and compete in. Back before Nike, Adidas, Brooks, and other companies became prevalent in the track and field scene, the footwear that the athletes used were more simple with less focus on how the shoe can help the athlete perform. This approach ultimately leads to many athletes being prone to injury and we can observe that today in current conditions when athletes get injured from wearing the wrong type of shoes for a prolonged period of time. Unfortunately, the data and analysis just wasn't prevalent at the origins of the sport which could be a reason for why times and marks in events improved drastically as big data and technology improved over time. Another area that benefited from the introduction of big data is overall understanding of fitness. There were plenty of superstitions and theories when it came to training for track athletes, but the aggregation of big data wasn't there in order to take into consideration the intensity and volume a track athlete's training should be at to push their body to the limit. In addition to these areas, another disadvantage in the era before big data alongside the lack of technology in general was that it was a difficult and cumbersome process to get meet results and use that in a productive way. Before big data, ways to gather masses of results to compare and rank individuals quickly and easily just did not exist. This clearly made analysis and prediction much more difficult than it is now and most likely caused a fair amount of confusion, especially at a lower, more unstructured level such as high school track and field. The list of inefficiencies could go on and on when talking about the life in track and field before big data analytics. There are certainly still some imperfections in the sport today, but the impact of big data has numerous positive effects.

\section{The Impact of Big Data on Track and Field}
As big data started to have an impact on the entire sports world, track and field saw many advancements in various forms such as shoe advancement, understanding fitness, form analysis, and result aggregation. With all of these aspects combined into one package, the sport of track and field not only saw improvements in times and marks in running and field events, but we can see where big data has benefited with the health and injury prevention of athletes as well.

\subsection{Shoe Advancement}
One of the most interesting and complex portions of current day track and field, especially in distance running, is the shoe development and the engineering that goes into the production between many brands and models. Whether it be Nike, Brooks, Adidas, Asics, or any of the other shoe companies with a serious stake in the running shoe market, the main goal is to provide the athlete with a shoe that enhances performance in a comfortable and efficient manner. This has led each company to come up with various technologies of their own using big data and analysis over time. Much of what is done and the many failures we will never see and this is perfectly modeled by what Nike recently achieved in their attempt to prepare athletes such as Eliud Kipchoge, Zersenay Tadese, and Lelisa Desisa to break the two hour barrier in the marathon by creating special shoes and modifying training based on data collected and analyzed. In Nike's own words on their website, ``during what was called Camp One, we brought these three Breaking2 runners to Nike for extensive testing, gathering data to guide the development of their respective shoes'' \cite{Innovation}. Here, we get a brief look behind the closed doors at Nike's special facility for shoe testing. After collecting data and running tests, they successfully created one of the most controversial and fascinating running shoes in years. They call it the Nike Zoom Vaporfly Elite and is only given to elite athletes sponsored by Nike, but later released a few modified versions to sell to the masses based on more testing. Of course, the data collected was based on three runners in this case, but these companies make many more shoes adhere to the common man and woman's needs based on much more data and testing. This, however, shows the willingness of a company to put in the time in order to put big data analysis to the use in order to make shoes that propel the sport forward to new heights.

\subsection{Understanding Fitness and Training}
Aside from making sure that track and field athletes are training in the right shoes, priming the body with the correct training is another area in which big data is making progress much easier. If you can't prepare your body to go the distance, jump farther or higher, and throw further then fancy shoes with a lot of tech are not of much use. Fortunately, with the rise of smart watches in athletics, there is an abundance of biometric data being collected on a constant basis. Many other procedures such as the VO2 max test and more are used to collect data based on any given person's capacity for endurance activities, but the use of technology such as a smart watch can give us instant and remote access into an athlete's biometric data over many exercises which can clearly tell a much bigger picture. In August of 2017, Business Insider published an article titled ``Here's how people are using their smartwatches'' which gave some insight into what most people use their smart watches for. Although the main usage was for notifications and text, they found that activity tracking is second leading function that users utilize \cite{Cakebread2017}. Although these results do not cover track and field athletes specifically, athletes are much more likely to use these functions if a coach requires it. This aggregation of data collected by a wearable device like a smart watch allows for coaches, physical therapists, and others to understand how a track and field athlete's body responds under different circumstances. Big data analysis being used in this way allows for better training plans based on athlete fitness which leads to healthier and safer training overall.

\subsection{Form Analysis}
On top of the rise of wearable devices, technological advances in track and field also came in the form of better video capture that provided more big data to analyze. Form analysis and improvement is another aspect of track and field that has benefited greatly from big data analytics. Recording video of an athlete's performance in various events from running to throwing is important in analyzing form to find areas that have room for improvement in technique. A video analysis and data tracking tool called Hudl has made big changes in how we approach film in sports in general. Track and Field coaches all over from the high school to college level and beyond are utilizing this tool for many things including form analysis. In 2016, Fast Company published an article titled ``How Hudl’s Mobile-Video Software Is Transforming Sports''. They assert that technology such as ``wearable technologies, high-speed cameras, Doppler radar, and data-collection devices'' exist to allow for the measurement of many complex movements and techniques \cite{Shaer2016}. Coaches across all sports and disciplines already take advantage of this technology and it has been a driving force in assisting track and field coaches give proper feedback on form and technique correction that their athletes can make for better results. In some events such as long jump, high jump, and pole vault, tweaking form can lead to drastic improvement in a short time.

\subsection{Meet Results Aggregation and Athlete/Team Rankings}
Another current day impact that big data has had on the sport of track and field is the improved aggregation of results in a useful and meaningful manner. With track meets and competitions happening every day in locations all over the world, having a place collect all of these results and sort them out is a game changer. It allows for quick analysis and comparison of performances that help rank the best athletes and teams among a nation. This is exactly what tfrrs.org tackles and handles well for United States college track and field and cross country across all divisions. The website collects results from all over the nation and displays it in a well organized manner while using the individual and team results to calculate rankings of the best individuals and teams in the nation. Progress like this is what drives competition and improvement. It allows for athletes, coaches, and fans to make simple comparisons to have more awareness of competition at a larger scale.

\section{Conclusions}
Looking back on the major impacts that big data has had on a sport most people gloss over such as track and field, it is difficult to imagine sports living on without big data. In many ways, track and field has made exponential growth in recent years thanks to big data analytics. We often take for granted things like the shoes on the shelves, access to useful smart watches, and websites that simplify our lives, but taking a deeper look should bring about some appreciation for what big data can be responsible for. Sometimes, we see big data as no more than a buzz word to gravitate attention towards something new and fascinating, but here we see the power it has to make a difference. Although, there is always room for improvement in any field and that is still the case here. A lot of the advancements we discussed were fairly recent and could most certainly lead to bigger and better conventions down the road. In the world of track and field, everyone is always looking for more tools and strategies to get faster and stronger. Not only have we made a case for strong involvement of big data in the growth of track and field, but we can make a case that big data should have a bigger role in every aspect of the sport going forward.

\begin{acks}
The author would like to thank Professor Gregor von Laszewski for providing the opportunity to explore a topic of deep interest.

\end{acks}


\bibliographystyle{ACM-Reference-Format}
\bibliography{report} 

\end{document}
