\documentclass[sigconf]{acmart}

\usepackage{graphicx}
\usepackage{hyperref}
\usepackage{todonotes}

\usepackage{endfloat}
\renewcommand{\efloatseparator}{\mbox{}} % no new page between figures

\usepackage{booktabs} % For formal tables

\settopmatter{printacmref=false} % Removes citation information below abstract
\renewcommand\footnotetextcopyrightpermission[1]{} % removes footnote with conference information in first column
\pagestyle{plain} % removes running headers

\newcommand{\TODO}[1]{\todo[inline]{#1}}

\begin{document}
\title{Big Data Analytics in Sports - Track and Field}


\author{Mathew Millard}
\affiliation{%
  \institution{Indiana University Bloomington}
  \city{Bloomington} 
  \state{Indiana} 
  \country{USA}
}
\email{mdmillar@indiana.edu}

\renewcommand{\shortauthors}{M. Millard}

\begin{abstract}
Over the years, sports analytics have become more prominent in the way sports are evolving. Data is the driving force in how these evolutions and changes are being made. With the presence of Big data and the impact it has on how sports decisions are being made, the importance on how this data can be used and how it can be presented in a useful manner have become a focal point for furthering athletics. Big data analytics within sports can cover a broad spectrum of topics, but the focus here is to dive deeper into the sport of track and field.
\end{abstract}

\keywords{i523, hid216, Track and Field, Big Data Analytics, Sports}


\maketitle

\section{Introduction}
When people think about big data analytics in sports, many think about sports such as basketball, baseball, and football because of the popularity and the volume of statistics used in those sports. Track and field, however, can benefit greatly from the same treatment that other sports have received. Although the sport of track and field is not a widely popular sport, there are many who take part in the sport due to the accessibility of being involved. Events on the running side of this sport range from the one hundred meter dash all the way up to the longer distance races such as the ten thousand meter run with many different types of races in between. On the field event side, there are various events including long jump, high jump, javelin, and many more. With all of these different events, there is a strong sense of specialization among athletes. An attention to detail and technique in the many facets of the sport is what drives the need for data analytics and statistical analysis.

\section{The Body of The Paper}
This is the body of my paper

\section{Conclusions}
This is my conclusion

\begin{acks}
The author would like to thank Professor Gregor von Laszewski for providing the opportunity to explore a topic of deep interest.

\end{acks}

\bibliographystyle{ACM-Reference-Format}
\bibliography{report} 

\end{document}
