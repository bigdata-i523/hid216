\documentclass[sigconf]{acmart}

\usepackage{hyperref}

\usepackage{endfloat}
\renewcommand{\efloatseparator}{\mbox{}} % no new page between figures

\usepackage{booktabs} % For formal tables

\settopmatter{printacmref=false} % Removes citation information below abstract
\renewcommand\footnotetextcopyrightpermission[1]{} % removes footnote with conference information in first column
\pagestyle{plain} % removes running headers

\begin{document}
\title{Big Data Analytics in Sports - Track and Field}


\author{Mathew Millard}
\affiliation{%
  \institution{Indiana University Bloomington}
  \streetaddress{938 N Walnut St. Apt. G}
  \city{Bloomington} 
  \state{Indiana} 
  \postcode{47404}
}
\email{mdmillar@indiana.edu}


\begin{abstract}
This paper covers the impact that Big Data has and could have on the sport of track and field.
\end{abstract}

\keywords{ACM proceedings, \LaTeX, text tagging}


\maketitle

\section{Introduction}

The \textit{proceedings} are the records of a
conference. ACM seeks to give these
conference by-products a uniform, high-quality appearance.  To do
this, ACM has some rigid requirements for the format of the
proceedings documents: there is a specified format (balanced double
columns), a specified set of fonts (Arial or Helvetica and Times
Roman) in certain specified sizes, a specified live area, centered on
the page, specified size of margins, specified column width and gutter
size.

\section{The Body of The Paper}



\section{Conclusions}


\bibliographystyle{ACM-Reference-Format}
\bibliography{report} 

\end{document}
